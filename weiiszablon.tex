% typ dokumentu
\documentclass[12pt,twoside]{article}

% użycie pakietu , jak include
\usepackage{weiiszablon}

% autor pracy
\author{Krystian Olszowy}

% np. EF-123456, EN-654321, ..., Numer albumu
\studentID{EA-167582}

\title{Aplikacja na smartfony do sterowania telewizorem}
\titleEN{{Mobile application for controlling TV set}}


%%% wybierz rodzaj pracy wpisując jeden z poniższych numerów: ...
% 1 = inżynierska	% BSc
% 2 = magisterska	% MSc
% 3 = doktorska		% PhD
%%% na miejsce zera w linijce poniżej
\newcommand{\rodzajPracyNo}{1}


%%% promotor
\supervisor{dr inż. Jan Sadolewski}
%% przykład: dr hab. inż. Józef Nowak, prof. PRz

%%% promotor ze stopniami naukowymi po angielsku
\supervisorEN{DEng Jan Sadolewski}

\abstract{Treść streszczenia po polsku}
\abstractEN{Treść streszczenia po angielsku}

\keywords{max. 5 słów kluczowych w 2 wierszach, oddzielanych przecinkami}
\keywordsEN{max. five keywords in English}


\begin{document}

% strona tytułowa
\maketitle

\blankpage

% spis treści
\tableofcontents

\clearpage
\blankpage


\section*{Wykaz symboli, oznaczeń i skrótów (opcjonalny)}
\addcontentsline{toc}{section}{Wykaz symboli, oznaczeń i skrótów (opcjonalny)}%

1 $\div$ 2 stron wykaz ważniejszych symboli i oznaczeń (jeśli jest potrzebny).
\clearpage

\section{Wstęp}
Być może punkty nie powinny być wyszczególnione?
\subsection{Sposoby sterowania telewizorem}
Ogólny wstęp o sposobach sterowania telewizorem i sygnalizacja kolejnych podpunktów.
\subsection{Płytki rozwojowe w szybkim prototypowaniu}
Krótka syganlizacja słuszności wyboru sposobu budowy prototypu.
\subsection{Smartfon i jego powszechność}
Kolejna sygnalizacja słuszności wyboru, tym urządzenia służącego za pilota.
Opis jak jest powszechny i prosty w użyciu a tym samym aplikacja jest odpowiednim wyborem.
\subsection{Metody komunikacji aplikacji mobilnych z urządzeniami wbudowanymi}
Bardzo krótkie wspomnienie o dostępnych rozwiązaniach i deliaktne wskazanie zalet BLE.

\clearpage
\section{Porównanie zaprojektowanego systemu z dostępnymi rozwiązaniami}
\subsection{Dostępne rozwiązania}
Wyliczenie i opisanie zalet i wad najważniejszych pozycji.
\subsection{Zestawienie z autorskim projektem}
Ogólna opinia na temat pól przewagi autorskiego rozwiązania.
\clearpage
\section{Przedstawienie wykorzystanych technologii}
W każdym z rodziałów opis teoretyczny zagadniena i uzasadnienie wyboru.
\subsection{Platforma ESP32}
\subsection{Język programowania C/C++}
\subsection{Framework Arduino}
\subsection{Komunikacja przez podczerwień}
\subsection{Android}
\subsection{Język programowania Dart}
\subsection{Framework Flutter}
\subsection{Komunikacja przez Bluetooth Low Energy}

\clearpage	

\section{Prototyp urządzenia wysyłającego sygnały IR}
\subsection{Schemat elektryczny zaprojektowanego urządzenia}
Opiszę sposób połaczenia wraz z rysunkiem i zdjęciem oraz opisem podjętych decyzji itp.
\subsection{Przegląd i opis użytych komponentów}
Zcharateryzuję wszystkie użyte fizyczne komponenty(nie jest ich wiele).
\clearpage

\section{Oprogramowanie sterujące urządzeniem wysyłającym sygnały IR}
\subsection{Ogólna struktura oprogramowania sterującego}
Opis całej struktury oprogramowania i przejście przez wybrane elementy pliku main.cpp.
\subsection{Moduł komunikacji przez BLE}
Opis modułu odpowiedzialnego za komunikację przez Bluetooth.
\subsection{Moduł interpretujący dane z BLE}
Opis autorskiego modułu interpretującego 2 charakterystyki serwera BLE.

\subsection{Biblioteka obsługi ekranu OLED}
Opis autorskiego modułu-nakładki do wyświetlania elementów na ekranie OLED.
\clearpage

\section{Aplikacja sterująca telewizorem}
\subsection{Struktura aplikacji}
Przedstawienie ogólnego zamysłu aplikacji i podziału na ekrany.
\subsection{Moduł obsługujący BLE}
Opis modułu odpowiedzialnego za komunikację BLE.
\subsection{Ekran połączenia z urządzeniem wysyłającym sygnał IR}
Przedstawienie funkcjonalności ekranu i przyjętych rozwiązań.
\subsection{Moduł obsługi przechowywania modelu telewizora}
Przedstawienie tego modułu.
\subsection{Ekran wyboru i edycji modelu telewizora}
Przedstawienie funkcjonalności ekranu i przyjętych rozwiązań.
\subsection{Ekrany przycisków pilota}
Przedstawienie funkcjonalności ekranu i przyjętych rozwiązań.
\clearpage

\section{Testy i prezentacja finalnego systemu}
W każdym z rozdziałów ewentualne wady rozwiązania i komentarze.
\subsection{Ustanawianie połączenia aplikacji z urządzeniem}
Przedstawienie sposobu nawziązywania połączenia aplikacji mobilnej z mikrokontrolerem.
\subsection{Edycja i wybór modelu telewizora w aplikacji}
Przedstawienie dostępnych opcji dodania, usunięcia i edycji czy wyboru modelu telewizora w którym znajdują się przyciski z kodami IR.
\subsection{Korzystanie z przycisków pilota}
Prezentacja działania systemu w kontakcie z telewizorem.

\clearpage

\section{Zakończenie}

1 $\div$ 3 stron merytorycznie podsumowanie najważniejszych elementów pracy oraz wnioski wynikające z osiągniętego celu pracy. Proponowane zalecenia i modyfikacje oraz rozwiązania będące wynikiem realizowanej pracy.

Ostatni akapit podsumowania musi zawierać wykaz własnej pracy dyplomanta i zaczynać się od sformułowania: „Autor za własny wkład pracy uważa: \ldots”.

\clearpage

\section*{Załączniki}
\addcontentsline{toc}{section}{Załączniki}

Według potrzeb zawarte i uporządkowane uzupełnienie pracy o dowolny materiał źródłowy (wydruk programu komputerowego, dokumentacja kons\-truk\-cyj\-no-\-tech\-no\-lo\-gicz\-na, konstrukcja modelu -- makiety -- urządzenia, instrukcja obsługi urządzenia lub stanowiska laboratoryjnego, zestawienie wyników pomiarów i obliczeń, informacyjne materiały katalogowe itp.).


\clearpage

\addcontentsline{toc}{section}{Literatura}

\begin{thebibliography}{4}
\bibitem{str} http://weii.portal.prz.edu.pl/pl/materialy-do-pobrania. Dostęp 5.01.2015.
\bibitem{Jakubczyk1997} Jakubczyk T., Klette A.: Pomiary w akustyce. WNT, Warszawa 1997.
\bibitem{Barski2011} Barski S.: Modele transmitancji. Elektronika praktyczna, nr 7/2011, str. 15-18.
\bibitem{dokum} Czujnik S200. Dokumentacja techniczno-ruchowa. Lumel, Zielona Góra, 2001.
\bibitem{Pawluk2001} Pawluk K.: Jak pisać teksty techniczne poprawnie, Wiadomości Elektrotechniczne, Nr 12, 2001, str. 513-515.
\end{thebibliography}

\clearpage

\makesummary

\end{document} 
