% typ dokumentu
\documentclass[12pt,twoside,draft]{article}

% użycie pakietu , jak include
\usepackage{weiiszablon}

% autor pracy
\author{Krystian Olszowy}

% np. EF-123456, EN-654321, ..., Numer albumu
\studentID{EA-167582}

\title{Aplikacja na smartfony do sterowania telewizorem}
\titleEN{{Mobile application for controlling TV set}}


%%% wybierz rodzaj pracy wpisując jeden z poniższych numerów: ...
% 1 = inżynierska	% BSc
% 2 = magisterska	% MSc
% 3 = doktorska		% PhD
%%% na miejsce zera w linijce poniżej
\newcommand{\rodzajPracyNo}{1}


%%% promotor
\supervisor{dr inż. Jan Sadolewski}
%% przykład: dr hab. inż. Józef Nowak, prof. PRz

%%% promotor ze stopniami naukowymi po angielsku
\supervisorEN{Jan Sadolewski, PhD, Eng.}

\abstract{Treść streszczenia po polsku}
\abstractEN{Treść streszczenia po angielsku}

\keywords{max. 5 słów kluczowych w 2 wierszach, oddzielanych przecinkami}
\keywordsEN{remote controller, BLE, IR, Flutter, mobile application}


\begin{document}

% strona tytułowa
\maketitle

\blankpage

% spis treści
\tableofcontents

\clearpage
\blankpage

\section{Wstęp}
\subsection{Telewizor i sposoby sterowania nim}
{Współcześnie nie wyborażamy sobie domu, w którym nie ma telewizora. Telewizory, będące 
centralnym elementem więszkości domostw, przekształciły się z tradycyjnych urządzeń 
telewizyjnych w zaawansowane systemy multimedialne, oferujące nie tylko dostęp do programów telewizyjnych,
ale również do różnorodnych treści wideo, gier, czy nawet aplikacji internetowych. 
Ich powszechność we współczesnym świecie, a także mnogość producentów i ich pomysłów 
sprawiło, że wyodrębniło się wiele technologii i sposobów sterowania tymi urządzeniami.

W wielu gospodarstwach domowych znajduje się wciąż telewizor sterowany jedynie za pomocą podczerwieni i jest
 to nadal najpopularniejszy sposób sterowania tymi urzadzeniami. Rozwiązania takie jak sterowanie poprzez WiFi, czy Bluetooth
 mimo, że są już pewien czas na rynku jeszcze nie goszczą u wszystkich użytkowników, zazwyczaj z powodu ceny
 nowego urządzenia multimedialnego lub zwykłego braku potrzeby wymiany telewizora. Czy jest jednak sposób aby
móc korzystać z wygody nowych rozwiązań komunikacji bezprzewodowej, mając urządzenie zdolne do sterowania 
telewizorem zawsze w naszej kieszeni dla urządzeń telewizyjnych nie posiadających obsługi nowoczesnych technologii bezprzewodowych?
}

\subsection{Smartfon i jego powszechność}
{W dynamicznie rozwijającym się świecie, smartfony stały się nieodłącznym elementem życia społecznego, 
definiując nowy wymiar komunikacji, rozrywki i funkcji użytkowych. 
W ciągu ostatnich kilku dekad, powszechność smartfonów osiągnęła niespotykany poziom, 
stając się nie tylko przedmiotem codziennego użytku, lecz również nieodzownym 
narzędziem wspierającym wszystkie aspekty naszego życia.

Obecnie smartfony są nie tylko urządzeniami komunikacyjnymi, ale także mobilnymi centrami multimedialnymi,
 umożliwiającymi dostęp do rozmaitych treści, od filmów i muzyki, po aplikacje społecznościowe. 
 Ich interaktywne interfejsy i intuicyjne systemy operacyjne sprawiają, że stają się łatwe w obsłudze nawet dla osób starszych, 
 które powoli przekonują się do współczesnej wersji telefonów komórkowych.

 Z uwagi na wyżej wymienione zalety samrtfonów naturalnym pomysłem wydaje się także użycie ich do 
 sterowania urządzeniami multimedialnymi, w tym także telewizorów. Właściwie wszystkie współczesne smartfony posiadają 
 wcześniej wspomniane nowoczesne technologie komunikacji bezprzewodowej jak WiFi i Bluetooth, jednak nie uświadczymy
 zbyt często już w nich obsługi podczerwieni. Aby skomunikować więc przy ich pomocy smartfon
 ze starszym telewizorem, powstaje potrzeba dołączenia pośrednika tłumaczącego interakcję 
 użytkownika z aplikacją mobilną na sygnały podczerwone, które zrozumie telewizor.}

\subsection{Cel pracy}
Celem pracy jest zaproponowanie rozwiązania sterowania telewizorem z poziomu 
smartfona dzięki aplikacji mobilnej, komunikującej się z systemem pośredniczącym opartym o płytkę , który to system ma być oparty o mikrokontroler wraz odpowiednimi komponentami, mający za zadanie wysyłać sygnały podczerwone do telewizora. System ma zapewniać możliwość programowania przycisków dzięki dedykowanemu ekranowi w aplikacji i 
wyświetlania odebranego kodu sygnału podczerwonego na wyświetlaczu zintegrowanym z mikrokontrolerem.
 
\subsection{Zakres pracy}
Zakres pracy obejmuje omówienie podobnych dostępnych rozwiązań na rynku, budowę systemu z~mikrokontrolerem na płytce ESP32 razem z wymaganymi modułami i utworzenie jego oprogramowania sterującego ,~a także projekt i oprogramowanie aplikacji
mobilnej frameworka Flutter służącej jako interfejs pilota uniwersalnego do telewizora. W obszarze zagadnień pedejmowanych w pracy znajduje się także przedstawienie działania zaprojektowanej aplikacji współpracującej ze zbudowanym urządzeniem pośredniczącym oraz wskazanie potencjalnych możliwości rozbudowania systemu.

\subsection{Zawartość pracy}
W rozdziale drugim  omówiono ogólnodostępne rozwiązania stanowiące aktualny stan wiedzy w zakresie zdalnego sterowania odbiornikiem telewizyjnym.

\clearpage
\section{Porównanie zaprojektowanego systemu z do\-stęp\-ny\-mi rozwiązaniami}
\subsection{Omówienie dostępnych rozwiązań}
{Na rynku można znaleźć niezliczoną ilość pilotów uniwersalnych. Rożnią się one 
ceną, obsługiwanymi funkcjonalnościami czy sposobami zasilania. Większość z nich zazwyczaj 
wykorzystuje programowanie oparte na nasłuchiwaniu sygnałów, co dla niektórych użytkowników może stanowić 
wyzwanie ze względu na uciążliwą obsługę procesu, wymagającą ciągłej modyfikacji odbieranego 
sygnału podczerwonego dla każdego przycisku. Cena rzędu jedynie kilku czy kilkunastu złotych 
już na polskim rynku jednak kusi potencjalnych nabywców pomimo wątpliwej jakości wykonania. Przykładem 
takiego pilota uniwersalnego może być urządzenie Interlook L336\cite{cheapController} dostępne w serwisie allegro.pl. 

Więksi producenci oferują również piloty uniwersalne z predefinowanymi sygnałami dla przycisków jednak
prawie zawsze są to modele właśnie tego producenta. Oczom nie umyka tażke wyższa cena takich
rozwiązań argumentowana zazwyczaj prestiżem marki i jakością wykonania. Takie rozwiązanie znaleźć
możemy na przykład w modelu Philips SRP4030/10\cite{expensiveController} możliwym do nabycia w x-kom.pl.

Dostępne są jednak także aplikacje z dedykowanymi urządzeniami wysyłającymi sygnały podczerwone, 
podłączanymi poprzez port microUSB w smartfonie. Znajdujemy już w nich wiele z pożądanych cech w pilocie
 uniwersalnym jak: niska cena, możliwość otrzymania predefiniowanych modeli telewizorów 
 czy edycję nie wymagającą posiadania oryginalnego pilota do sterowanego urządzenia. Przykładowym dostępnym
rozwiązaniem tego typu jest model urządzenie EKX4S-T\cite{appController}} dostępny na allegro.pl.
\subsection{Zestawienie z autorskim projektem}
{Wymienione wcześniej urządzenia mają swoje zastosowania jedank nie wystrzegają się wad. 
Wiele z nich wymaga zasilania bateryjnego co jest szkodliwe dla środowiska, o które szczególnie
stara się dbać społeczeństwo w dzisiejszych czasach. Posiadają one często mozolne i łatwe w niepoprawnym użyciu 
programowanie. Każde z tych urządzeń lub ich komponentów można łatwo zgubić z uwagi na sposób 
ich używania. Trzymanie w dłoniach czy wtykanie i wyciąganie komponentu ze smartfona znacznie
sprzyja zapodzianiu tych gadżetów. Podczas relaksu, kiedy nikt nie jest szczególnie 
ostrożny, zdecydowanie łatwiej o takie nieszczęście. Dostęp do tworzenia własnych 
zestawów przycisków jest w nich nierzadko trudny, czasem niemożliwy, a użycie często wymaga 
przeskakiwania po liście i sprawdzeniu czy telewizor reaguje na predefiniowane przyciski.

Autorskie rozwiązanie systemu służacego za pilota do telewizora stara się rozwiązać te problemy. Bateria czy akumulator możliwe 
są do używania z urządzeniem wysyłającym sygnały podczerwone, jednak z powodzeniem zasilane może i powinno odbywać się przez 
kabel microUSB, dzięki czemu nie zwiększany jest popyt na szkodliwe dla środowiska baterie. Programowanie przycisków
odbywa się przez podanie kodów IR w odpowiednich polach, które sprawdzją czy są one poprawne. Utworzony przez użytkownika lub predefiniowany
 model telewizora wybierany jest z listy, którą można łatwo przeglądać i rozbudować. Sama aplikacja posiada kontrolę błędów
 i intuicyjnie przeprowadza korzystającego przez jej funkcję w kierunku czytania.
}
\clearpage
\section{Przedstawienie wykorzystanych technologii}
\subsection{Platforma ESP32[obraz płytki]}
ESP32 jest układem typu SoC (ang. System-on-a-chip) produkowanym przez chińską firmę Espressif Systems, 
wyposażony jest najczęściej w dwurdzeniowy procesor Xtensa LX6 o taktowaniu 240 MHz. W większości przypadków używa się
tego układu osadzonego na płytce deweloperskiej co przyspiesza prototypowanie i projektowanie rzeczywistego 
systemu. Na runku znajaduje sie niezliczona ilość różnych wersji płytek deweloperskich przez co dopasowanie
jej do własnych potrzeb jest znacznie łatwiejsze niż rozważając inne rozwiązania.

Jednym z kluczowych atutów ESP32 jest wbudowane wsparcie dla Wi-Fi(802.11 b/g/n) i Bluetooth(klasyczny 4.2 i BLE),
 co umożliwia łatwe połączenie z siecią bezprzewodową oraz sprawną komunikację między urządzeniami zewnętrznymi.
 Duża licz\-ba portów wejścia/wyjścia czyni go idealnym do obsługi wielu czujników, akscesoriów i modułów. Układ zawiera także 
zintegrowany konwerter analogowo-cyfrowy (ADC) umożliwia precyzyjny pomiar sygnałów analogowych.

ESP32 został także zoptymalizowany pod kątem efektywnego zużycia energii, co sprawia, że jest bardzo dobrym
 wyborem dla aplikacji zasilanych bateryjnie lub kiedy zwyczajnie ważny jest niski pobór prądu. Jego wszechstronność objawia
się obsługą różnych protokołów komunikacyjnych używanych głównie dla przewodowej komunikacji, 
takich jak SPI, I2C, UART, co ułatwia integrację z zewnętrznymi modułami i systemami.

Warto podkreślić, że ESP32 cieszy się uznaniem wśród społeczności deweloperów, co przekłada się na dostęp
 do obszernej dokumentacji oraz wsparcie online. Programowalność w języku Arduino oraz możliwość korzystania 
 z MicroPython dodatkowo zwiększają elastyczność platformy, umożliwiając dostosowanie do indywidualnych potrzeb projektu.
 Nieocenione są także biblioteki i moduły tworzone przez społeczność i udostępniane w formie otwartoźródłowej.
\subsection{Język programowania C++}
Język C++ \cite{swider2004} oferuje zaawansowane funkcje i umożliwia programistom skuteczną pracę nad różnorodnymi projektami związanymi z systemami wbudowanymi.
W tym kontekście, kluczową cechą języka C++ jest jego zdolność do zapewniania niskopoziomowej kontroli nad sprzętem,
co jest kluczowe w przypadku urządzeń o ograniczonych zasobach. 

Język ten umożliwia także stosowanie obiektowego podejścia do projektowania, co z kolei sprzyja modularności i łatwości utrzymania kodu.
Tworzenie klas i obiektów pozwala na strukturyzację kodu, co jest istotne w przypadku skomplikowanych systemów wbudowanych, gdzie 
czytelność i zrozumienie kodu są kluczowe dla sukcesu projektu.

C++ daje także możliwość korzystania z funkcji inline, co przekłada się na minimalizację narzutu czasowego związanego z wywołaniami funkcji.
To istotne, gdy każdy cykl procesora ma znaczenie, a efektywność kodu jest priorytetem. Ważną cechą tego języka  
 jest również możliwość korzystania z mechanizmów zarządzania pamięcią, takich jak alokacja dynamiczna, co zapewnia elastyczność rozwiązania
 w zmiennych warunkach pracy systemu.
\subsection{Framework Arduino}
{Framework Arduino to otwarte oprogramowanie, które umożliwia łatwe tworzenie aplikacji dla mikrokontrolerów. 
Bazuje na języku programowania C++ oraz wykorzystuje uproszczoną warstwę abstrakcji, co sprawia, że 
jest przyjazny dla początkujących, jednocześnie oferując zaawansowane funkcje dla doświadczonych programistów.
Oferuje wiele gotowych bibliotek i narzędzi, które ułatwiają programowanie mikrokontrolerów, 
co jest szczególnie istotne w kontekście prototypowania i szybkiego tworzenia projektów.
Sama składnia języka Arduino jest uproszczona i zbliżona do języka C++ wraz z jego podejściem obiektowym.

Arduino oferuje także prostą obsługę wejścia/wyjścia (I/O), dzięki czemu integracja z czujnikami, przekaźnikami,i wielomai 
innymi modułami jest błyskawiczna i intuicyjna. Ponadto, framework zapewnia prosty sposób dostęp do interfejsów komunikacyjnych,
 takich jak UART, SPI czy I2C. Ważną częścią frameworka Arduino są też przyjazne w użyciu biblioteki. Obejmują one obszar komunikacji 
 bezprzewodowej, obsługi czujników, sterowania silnikami, czy obsługi ekranów LCD.}
\subsection{Komunikacja za pomocą podczerwieni[opisać komunikację NEC, przykład przesłania kodu]}
{Komunikacja przez podczerwień (IR) to forma przesyłania danych, wykorzystująca fale elektromagnetyczne
 o niższej częstotliwości niż światło widzialne. Fale podczerwone znajdują się w zakresie elektromagnetycznym 
 poniżej czerwonego końca widma światła widzialnego, typowo w zakresie od 300 GHz do 400 THz. Te fale są 
 niewidoczne dla ludzkiego oka, ale mogą być wykrywane i generowane przez diody. Typowo 
 wysyłając sygnał według konkretnego protokołu najpierw moduluje się go według określonego kodu liczbowego.
Modulacja może obejmować zmianę amplitudy, częstotliwości lub fazy fali podczerwonej. Gdy sygnał dociera do odbiornika 
(zazwycza zasięg wynosi do 10m), odbiornik wyposważony także w diodę interpretuje odebrany sygnał, przeprowadzając proces demodulacji, do kodu 
liczbowego. Z tak wysłanej informacji może teraz skorzystać urządzenie odbierające i podjąć na jej podstawie działania.}
\subsection{System operacyjny Android}
{System Android, rozwijany przez firmę Google, stanowi środowisko operacyjne oparte na otwartym kodzie źródłowym,
   co przyczynia się do jego wyjątkowej elastyczności i dostosowywalności do różnych urządzeń mobilnych, głównie 
   smartfonów i tabletów, ale z powodzeniem stosowany jest także inncyh urządzeniach przenośnych.
    Otwartość kodu umożliwia deweloperom z całego świata dostosowywanie systemu do specyficznych 
   potrzeb. Istotną cechą Androida jest także jego wszechstronność, manifestująca się w szerokiej gamie dostępnych 
   urządzeń, produkowanych przez różnych producentów co powoduje jednak, że wymagana jest obsługa w programowanej aplikacji różnych środowisk sprzętowych.
    System ten obsługuje różne poziomy cenowe urządzeń, umożliwiając 
   programistom tworzenie aplikacji dostępnych dla różnych grup odbiorców. Sklep Google Play, jako główne źródło 
   aplikacji, gier, multimediów i innych treści, stanowi istotny element ekosystemu Androida. Ponadto, integracja 
   z usługami Google, takimi jak Gmail, Google Drive czy Google Maps, ułatwia korzystanie z tych usług na urządzeniach 
   z systemem Android. Względną jednolitość tego środowiska sprawia, że dla programistów stanowi ono atrakcyjne pole do 
   rozwoju i tworzenia innowacyjnych rozwiązań, z uwagi na szeroką bazę użytkowników oraz dynamiczny rynek aplikacji mobilnych.}
\subsection{Język programowania Dart}
{Dart to język programowania stworzony przez Google, zaprojektowany z myślą o tworzeniu wydajnych, skalowalnych i nowoczesnych aplikacji. 
Jego głównym zastosowaniem jest budowa interaktywnych stron internetowych oraz aplikacji mobilnych przy użyciu frameworka Flutter.

Język ten charakteryzuje się statycznym typowaniem, co oznacza, że typy zmiennych są sprawdzane w trakcie kompilacji, co może pomóc 
w wykrywaniu błędów przed uruchomieniem programu. Dart wspiera również programowanie obiektowe,
 co umożliwia programistom organizację kodu w logiczne i zrozumiałe struktury.

Jedną z ważnych cech Darta jest również jego zdolność do wykonywania kodu zarówno w trybie just-in-time (JIT),
 jak i ahead-of-time (AOT). Tryb JIT umożliwia szybki rozwój i testowanie aplikacji, podczas gdy tryb AOT pozwala na kompilację
  kodu źródłowego do natywnego kodu maszynowego, co przyspiesza wydajność aplikacji podczas działania.

Dart oferuje także mechanizmy asynchroniczne, co jest istotne w programowaniu współbieżnym i 
obsłudze operacji wejścia/wyjścia bez blokowania głównego wątku.

W kontekście frameworka Flutter, Dart staje się kluczowym elementem do budowy interfejsów użytkownika. !!!!!!Błąd !!!!!Akapit ma conajmniej 2 zdania. }
\subsection{Framework Flutter}
{Flutter to open-source'owy framework stworzony przez Google do budowy interfejsów użytkownika (UI - User Inferface). 
Jest wykorzystywany do tworzenia aplikacji mobilnych, webowych i desktopowych, ze wskazaniem na aplikacje mobilne. 
Jednym z głównych atutów Fluttera jest możliwość tworzenia jednego kodu źródłowego, który może być używany budowany do
 aplikacji dla różnych platform, takich jak Android, iOS, web, Windows czy Linux.

W centrum Fluttera znajduje się język programowania Dart, który jest używany do definiowania interfejsu użytkownika oraz logiki biznesowej. 
Framework ten bazuje na podejściu deklaratywnym, co oznacza, że programiści opisują, jak ma wyglądać interfejs w danym momencie,
 a nie jak ma być aktualizowany w odpowiedzi na różne zdarzenia. To podejście ułatwia zrozumienie i utrzymanie kodu.
Wprowadza on również własny silnik renderujący. Dzięki temu aplikacje Fluttera często charakteryzują się płynnością i responsywnością.

Framework oferuje bogatą gamę wbudowanych widgetów, które są podstawowymi elementami budującymi interfejs użytkownika. 
Programiści mogą również tworzyć własne niestandardowe widgety, co umożliwia pełną swobodę w projektowaniu interfejsu.

Flutter wspiera hot-reloading, co pozwala na natychmiastowe obserwowanie zmian wprowadzanych w kodzie bez konieczności ponownego uruchamiania całej aplikacji.
To znacznie skraca czas pracy podczas rozwijania projektu.
Dzięki narzędziom takim jak Flutter DevTools, deweloperzy mają dostęp do zaawansowanych narzędzi do analizy, debugowania i optymalizacji swoich aplikacji.

Flutter zdobył popularność wśród programistów, zwłaszcza tych, którzy chcą tworzyć estetyczne, responsywne i wieloplatformowe 
aplikacje z minimalnym nakładem pracy co jest elementem chrakterystycznym dla tego rozwiązania.}
\subsection{Komunikacja przez Bluetooth Low Energy[jeśli dostępne jak to działa]}
{Bluetooth Low Energy (BLE) to technologia komunikacyjna zaprojektowana do efektywnej wymiany danych pomiędzy urządzeniami przy niskim zużyciu energii.
Komunikacja za pomocą BLE opiera się na koncepcji dwóch głównych typów urządzeń: urządzenia peryferyjne i centralne. 
Urządzenie peryferyjne emituje dane, podczas gdy urządzenie centralne zbiera te dane.

Cechą charakterystyczną BLE jest niskie zużycie energii, co sprawia, że jest idealne do zastosowań, gdzie ważne jest przedłużenie życia baterii w urządzeniach mobilnych.
 Komunikacja BLE opiera się na transmisji krótkich pakietów danych, co przyczynia się do efektywności energetycznej.

Protokół BLE definiuje dwie główne kategorie operacji: operacje bezpośrednie i operacje oparte na atrybutach.
 Operacje bezpośrednie umożliwiają natychmiastową wymianę danych, podczas gdy operacje oparte na atrybutach 
pozwalają na dostęp do atrybutów urządzenia, takich jak czujniki czy inne dane charakterystyczne.

Atrybuty w BLE reprezentują dane, które urządzenia wymieniają między sobą. Każdy atrybut ma swój identyfikator 
(UUID) i może przechowywać różne typy danych, takie jak liczby całkowite, tekst czy dane binarne.}

\clearpage	

\section{Prototyp urządzenia wysyłającego sygnały IR}
\subsection{Schemat elektryczny zaprojektowanego urządzenia}
Opiszę sposób połaczenia wraz z rysunkiem i zdjęciem oraz opisem podjętych decyzji itp.
\subsection{Przegląd i opis użytych komponentów}
Zcharateryzuję wszystkie użyte fizyczne komponenty(nie jest ich wiele).
\clearpage

\section{Oprogramowanie sterujące urządzeniem wysyłającym sygnały IR}
\subsection{Ogólna struktura oprogramowania sterującego}
Opis całej struktury oprogramowania i przejście przez wybrane elementy pliku main.cpp.
\subsection{Moduł komunikacji przez BLE}
Opis modułu odpowiedzialnego za komunikację przez Bluetooth.
\subsection{Moduł interpretujący dane z BLE}
Opis autorskiego modułu interpretującego 2 charakterystyki serwera BLE.

\subsection{Biblioteka obsługi ekranu OLED}
Opis autorskiego modułu-nakładki do wyświetlania elementów na ekranie OLED.
\clearpage

\section{Aplikacja sterująca telewizorem}
\subsection{Struktura aplikacji}
Przedstawienie ogólnego zamysłu aplikacji i podziału na ekrany.
\subsection{Moduł obsługujący BLE}
Opis modułu odpowiedzialnego za komunikację BLE.
\subsection{Ekran połączenia z urządzeniem wysyłającym sygnał IR}
Przedstawienie funkcjonalności ekranu i przyjętych rozwiązań.
\subsection{Moduł obsługi przechowywania modelu telewizora}
Przedstawienie tego modułu.
\subsection{Ekran wyboru i edycji modelu telewizora}
Przedstawienie funkcjonalności ekranu i przyjętych rozwiązań.
\subsection{Ekrany przycisków pilota}
Przedstawienie funkcjonalności ekranu i przyjętych rozwiązań.
\clearpage

\section{Testy i prezentacja finalnego systemu}
W każdym z rozdziałów ewentualne wady rozwiązania i komentarze.
\subsection{Ustanawianie połączenia aplikacji z urządzeniem}
Przedstawienie sposobu nawziązywania połączenia aplikacji mobilnej z mikrokontrolerem.
\subsection{Edycja i wybór modelu telewizora w aplikacji}
Przedstawienie dostępnych opcji dodania, usunięcia i edycji czy wyboru modelu telewizora w którym znajdują się przyciski z kodami IR.
\subsection{Korzystanie z przycisków pilota}
Prezentacja działania systemu w kontakcie z telewizorem.

\clearpage

\section{Zakończenie}

1 $\div$ 3 stron merytorycznie podsumowanie najważniejszych elementów pracy oraz wnioski wynikające z osiągniętego celu pracy. Proponowane zalecenia i modyfikacje oraz rozwiązania będące wynikiem realizowanej pracy.

Ostatni akapit podsumowania musi zawierać wykaz własnej pracy dyplomanta i zaczynać się od sformułowania: „Autor za własny wkład pracy uważa: \ldots”.

\clearpage

\section*{Załączniki}
\addcontentsline{toc}{section}{Załączniki}

Według potrzeb zawarte i uporządkowane uzupełnienie pracy o dowolny materiał źródłowy (wydruk programu komputerowego, dokumentacja kons\-truk\-cyj\-no-\-tech\-no\-lo\-gicz\-na, konstrukcja modelu -- makiety -- urządzenia, instrukcja obsługi urządzenia lub stanowiska laboratoryjnego, zestawienie wyników pomiarów i obliczeń, informacyjne materiały katalogowe itp.).


\clearpage

\addcontentsline{toc}{section}{Literatura}
\bibliography{biblgr}
\bibliographystyle{plain}
% \begin{thebibliography}{4}
% \bibitem{cheapController}{https://allegro.pl/oferta/pilot-uniwersalny-do-telewizora-tv-dvb-t-sat-dvd-11914593904}
% \bibitem{expensiveController}{https://www.x-kom.pl/p/1196443-pilot-uniwersalny-philips-pilot-do-tv-lg.html}
% \bibitem{appController}{https://allegro.pl/oferta/uniwersalny-adapter-pilota-na-podczerwien-14175757382}
% \end{thebibliography}

\clearpage

\makesummary

\end{document} 
